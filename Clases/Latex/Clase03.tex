%\documentclass{article}
\documentclass[etterpaper, 12pt, oneside]{article}%formatea el documento
\usepackage{amsmath}
\usepackage{graphicx}
\usepackage{enumitem}
\usepackage{xcolor}
\usepackage[utf8]{inputenc}
\graphicspath{{Imágenes/}}

%Aquí inicia la portada del documento
\title{\Huge Taller de herramientas computacionales}
\author{Fernando Moctezuma Soto}
\date{20/enero/2019}

\begin{document}
	\maketitle
	%includegraphics[scale=0.40]{1.jpg}
	
	\newpage
	
	\title{\Huge Notas del curso\\}
	
	\textbf{Clase del 9 de enero.}\\
	
	Al listar todos los documentos de una carpeta, con el comando \textbf{ls -al}, las carpetas contenidas comienzan con su nombre y a los archivos les antecede un guión. Los archivos ocultos comienzan con un punto.
	
	Los pasos generales para resolver un problema son:\\
	Definir: entender en qué consiste.\\
	Analizar y delimitar: los elementos que comprende.\\
	Soluciones: se contestan las preguntas ¿es posible solucionarlo?, ¿qué soluciones se han dado? y ¿cuáles son las herramientas para hacerlo?.\\
	Describir solución con detalle: el proceso con el que se soluciona el problema.
	
	

	
\end{document}
