\documentclass{beamer}
\usepackage{graphicx}
\usepackage[utf8]{inputenc}
\usepackage[spanish]{babel}
\graphicspath{{Imagenes/}}
\usetheme{Dresden} 
%Otros temas: Antibes, AnnArbor, CambridgeUS, Berkeley, Goettingen, Bergen, Hannover, Dresden, Ilmenau, Berlin, Boadilla, Darmstadt

%Esto unicamente se utiliza para el tema Bergen
\def\insertauthorindicator{¿Quién?}
\def\insertdateindicator{Fecha}
\title{Taller de herramientas computacionales}
\author{Fernando Moctezuma Soto}
\date{\today}

%\title{Taller de herramientas computacionales}
%\author{Fernando Moctezuma Soto}
%\date{22/enero/2019}

\begin{document}
	
	\maketitle
	
	
	\begin{frame}
	\transblindshorizontal
		\frametitle{Primer presentación en LaTeX}
	\end{frame}


	\begin{center}
		\includegraphics[scale=0.50]{1.jpg}
	\end{center}

	\begin{frame}
		\frametitle{Segunda diapositiva}
		Esta es la segunda diapositiva.
	\end{frame}

	\begin{frame}[fragile] %Para introducir carácteres especiales.
		\begin{verbatim}
			#!/usr/bin/python2.7
			# -*- coding: utf-8 -*-
			
			'''
			Moctezuma Soto, Fernando
			310126070
			Taller de herramientas computacionales
			
			Programa que imprime "Hoy es miércoles"
			'''
			
			
			x = 10.5;  y = 1.0/3;  z = 15.3
			# x, y, z = 10.5, 1.0/3, 15.3
			
			H = """El punto en R3 es:
			(x, y, z) = (%.2f, %g, %G)
			""" % (x, y, z)
			print H
			
			
			G = """
			El punto en R3 es:
			(x, y, z) = ({laX:.2f}, {laY:g}, {laZ:G})
			""" .format(laX=x, laY=y, laZ=z)
			print G
			
			
			
			import math as m
			from math import sqrt 
			#from math import *
			
			x=16
			x=input("¿Cuál es el valor al que le quieres calcular la raiz?: ")
			print "La raíz cuadrada de %.2f es %f" % (x, m.sqrt(x))
			print sqrt(16.5)
	
		\end{verbatim}
			
	\end{frame}

\end{document}