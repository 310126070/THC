%\documentclass{article}
\documentclass[etterpaper, 12pt, oneside]{article}%formatea el documento
\usepackage{amsmath}

\usepackage{graphicx}
\usepackage{enumitem}
\usepackage{xcolor}
\usepackage[utf8]{inputenc}
\graphicspath{{Imagenes/}}

%Aquí inicia la portada del documento
\title{\Huge Taller de herramientas computacionales}
\author{Fernando Moctezuma Soto}
\date{20/enero/2019}

\begin{document}
	\maketitle
	
	\begin{center}
		\includegraphics[scale=0.40]{1.jpg}
	\end{center}
	
	\newpage
	
	\title{\Huge Notas del curso\\}
	
	\textbf{Clase del 10 de enero.}\\
	

	
	Se analizó y delimitó el problema de la función:
	
	$y(t)=V_{0}t-\frac{1}{2}gt^{2}$
	
	 al ver el rango de valores que interesa evaluar, de acuerdo a la pregunta de cuáles son los errores de sintaxis y lógicos. Se emplean la siguientes fórmulas cuando interesa saber la posición después de aventar una pelota:
	
	$V_{0}t-\frac{1}{2}gt^{2}=0$
	
	$t(V_{0}-\frac{1}{2}gt)=0$
	
	$V_{0}-\frac{1}{2}gt=0$
	
	$t=2\frac{V_{0}}{g}$
	
	$t\epsilon[0, 2\frac{V_{0}}{g}]$
	
	Al precisar los límites del problema, los errores lógicos se identifican por los resultados del análisis y delimitación.
	
	Para definir un problema en particular se requiere asignar valores específicos, como los siguientes:
	
	$t=3$
	
	$V_{0}=4\frac{m}{s}$
	
	$g=9.81$
	
	pero, $[0, 0.81]$
	
	Para
	
	$V_{0}=34$
	
	$[0, 0.81]$
	
	es decir, se encuentra dentro del rango y
	
	$y(3)=(34)(3)-\frac{1}{2}(9.81)(3^{2})=57.85$\\
	
	Cuando se esta escribiendo el la terminal, se puede presionar una vez tabulador para autocompletar, y dos veces para mostrar una lista de posibilidades con la terminación de cierto comando.
	
	Los IDE tienen un editor especializado para el lenguaje, intérprete y herraminetas para la depuración.	Un programa en Python es un archivo de texto plano que contiene instrucciones válidas para Python. Para comenzar a trabajar en Python 2.7.15 Shell, desde la terminal se escribe \emph{idle}. Se puede abrir un nuevo archivo presionando \emph{New File} y una vez que se le introduzcan instrucciones, se ejecuta en el shell con \emph{F5}.
	
\end{document}
