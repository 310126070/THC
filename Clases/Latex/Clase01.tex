%\documentclass{article}
\documentclass[etterpaper, 12pt, oneside]{article}%formatea el documento
\usepackage{amsmath}
\usepackage{graphicx}
\usepackage{enumitem}
\usepackage{xcolor}
\usepackage[utf8]{inputenc}
\graphicspath{{Imagenes/}}

%Aquí inicia la portada del documento
\title{\Huge Taller de herramientas computacionales}
\author{Fernando Moctezuma Soto}
\date{7/enero/2019}

\begin{document}
	\maketitle
	%\includegraphics[scale=0.40]{1.jpg}
	
	\newpage
	
	\title{\Huge Notas del curso\\}
	
	\textbf{Clase del 14 de enero.}
	
	Se vieron los siguientes comandos. 
	
less: pagina el texto, y permite avanzar y retoceder en el archivo

man: muestra el manual del argumento introducido, que puede ser un programa, útil o función

set: el nombre y el valor de cada variable de shell

pwd: muestra la ruta de acceso del directorio actual

cd: cambia al directorio que se especifique, y en ausencia de éste cambia a /home

ls: muesta los nombres de los archivos del directorio actual por defecto, a menos que se especifique otro

df: muestra información sobre el sistema de ficheros

top: muestra la actividad en tiempo real del sistema

mkdir: crea uno o varios directorios si no existen

history: muestra la lista del historial de comandos con los números de línea

El sistema operativo es el programa que controla todos los recursos y proporcionar el soporte que necesitan los programas de aplicación. Es un programa que actúa de intermediario entre el usuario y el hardware del ordenador. Como ejemplo de sistemas operativos se pueden mencionar: UNIX, OS X y Microsoft.
		
		Los entornos de usuario proporcionan herramientas para acceder a la información, ejecutar aplicaciones, configurar el ordenador, etcétera. Los entornos de usuario son la interfaz que sirve para establecer la comunicación entre el usuario y el S.O. Existen entornos de usuario de línea de órdenes, donde estas se expresan como texto y de interfaz gráfico, donde las órdenes se expresan accionando sobre iconos visuales, botones y menús con un ratón. Entre los entornos de usuario se encuentran:
		
		 GNOME: Pretende construir un escritorio completo basado enteramente en software libre. Como gestor de ventanas suele usar Sawfish o Metacity. Está basado en las librerías GTK como las aplicaciones que incluye, ofreciendo un aspecto y comportamiento coherente. GNOME forma parte del proyecto GNU.
		 
		KDE: Proporciona una interfaz consistente para las aplicaciones X, tanto en apariencia como en funcionalidad. KDE contiene un conjunto básico de aplicaciones tales como un gestor de ventanas (kwm), gestor de archivos, emulador de terminal, sistema de ayuda y configuraciÓn de la pantalla. KDE especifica un toolkit GUI estándar así como unas librerías gráficas, las QT, que usan todas las aplicaciones. 
		
Entre las distibuciones de Linux más conocidas se encuentran Debian,que no la produce una empresa, sino una comunidad de voluntarios y RedHat con buena calidad en contenidos y soporte a los usuarios por parte de la empresa que la distribuye
	
\end{document}
