%\documentclass{article}
\documentclass[etterpaper, 12pt, oneside]{article}%formatea el documento
\usepackage{amsmath}
\usepackage{graphicx}
\usepackage{enumitem}
\usepackage{xcolor}
\usepackage[utf8]{inputenc}
\graphicspath{{Imágenes/}}

%Aquí inicia la portada del documento
\title{\Huge Taller de herramientas computacionales}
\author{Fernando Moctezuma Soto}
\date{16/enero/2019}

\begin{document}
	\maketitle
	%includegraphics[scale=0.40]{1.jpg}
	
	\newpage
	
	\title{\Huge Notas del curso\\}
	
	\textbf{Clase del 15 de enero.}\\
	
	Se distinguió entre una asignación y una comparación en Python. La asignación le da un valor a un símbolo (ej. a=5) y la comparación compara dos valores (ej. a==5).
	
	Se solucionó el problema de hallar el valor de la raíz de un número \emph x, y el de imprimir el número de veces que se repite un cierto proveso con \emph{while}.
	
	En Latex se comenzaron a ver las instrucciones para componer textos con fórmulas matemáticas.
	
	
\end{document}
