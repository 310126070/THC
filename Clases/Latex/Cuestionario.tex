%\documentclass{article}
\documentclass[etterpaper, 12pt, oneside]{article}%formatea el documento
\usepackage{amsmath}
\usepackage{graphicx}
\usepackage{enumitem}
\usepackage{xcolor}
\usepackage[utf8]{inputenc}
\graphicspath{{Imagenes/}}

%Aquí inicia la portada del documento
\title{\Huge Taller de herramientas computacionales}
\author{Fernando Moctezuma Soto}
\date{14/enero/19}

\begin{document}
	\maketitle
	
	\newpage
	
	\title{\Huge Cuestionario\\}

	\begin{enumerate}
		\item ¿Qué es un sistema operativo?
		\\R: Es el programa que controla todos los recursos y proporcionar el soporte que necesitan los de aplicación. Es un programa que actúa de intermediario entre el usuario y el hardware del ordenador.
	
		\item ¿Qué nombres sistemas operativos has escuchado?
		\\R: UNIX, OS X y Microsoft.
		
		\item ¿Para que sirven los entornos de usuario?
		\\R: Proporcionan herramientas para acceder a la información, ejecutar aplicaciones, configurar el ordenador, etcétera. Los entornos de usuario son la interfaz que sirve para establecer la comunicación entre
el usuario y el S.O. 

		\item ¿Cuáles son los entornos de usuario que hay?
		\\R: De línea de órdenes, donde estas se expresan como texto y de interfaz gráfico, donde las órdenes se expresan accionando sobre iconos visuales, botones y menús con un ratón.
		
		\item ¿Qué es un intérprte de ódenes o shell?
		\\R: Un programa que lee órdenes de teclado y controla su ejecución.

		\item ¿Qué nombres entornos de usuario has escuchado y cuáles son su características?
		\\R: GNOME: Pretende construir un escritorio completo basado enteramente en software libre. Como gestor de ventanas suele usar Sawfish o Metacity. Está basado en las librerías GTK como las aplicaciones que incluye, ofreciendo un aspecto y comportamiento coherente. GNOME forma parte del proyecto GNU.
		
KDE: Proporciona una interfaz consistente para las aplicaciones X, tanto en apariencia como en funcionalidad. KDE contiene un conjunto básico de aplicaciones tales como un gestor de ventanas (kwm), gestor de archivos, emulador de terminal, sistema de ayuda y configuraciÓn de la pantalla. KDE especifica un toolkit GUI estándar así como unas librerías gráficas, las QT, que usan todas las aplicaciones. 
		
		\item ¿Qué un sistema de ficheros?
		\\R: Es la representación los dispositivos de almacenamiento secundario que pueden conectarse a un ordenador: discos duros, otros discos externos o internos, CD-ROM, DVD, etc. 
		
		\item ¿Qué nombres distribuciones de GNU/Linux has escuchado?
		\\R: Debian, no la produce una empresa, sino una comunidad de voluntarios y RedHat con buena calidad en contenidos y soporte a los usuarios por parte de la empresa que la distribuye

		\item ¿Para que sirve un editor de textos?
		\\R: Para la edición de contenidos sin opciones de formato. Se utilizan habitualmente para escribir y editar programas.
		
		\item ¿Qué es una base de datos?
		\\R: Permite guardar grandes cantidades de información relacionada entre sí. Se definen las características de los datos, las formas de acceder a ellos y las relaciones que tienen entre sí.
		
		\item ¿Qué es Python?
		\\R: Python es un lenguaje de programación interpretado. El principal objetivo de Python es facilitar la lectura y diseño, y se desarrolla como un proyecto de código abiero.
		
		\item ¿Qué incluyen los IDE?
		\\R: Los IDE cuentan con un editor especializado para el lenguaje, intérprete y herramientas para depurar (buscar errores).
		
		\item ¿Qué es git?
		\\R: Para controlar las versiones en un sistema; registra los cambios realizados en un archivo o conjunto de archivos a lo largo del
tiempo, de modo que se pueda recuperar versiones específicas posteriormente.

		\item ¿Qué comandos se han utilizado en clase y cuáles son sus funciones?
		\\less: pagina el texto, y permite avanzar y retoceder en el archivo

man: muestra el manual del argumento introducido, que puede ser un programa, útil o función

set: el nombre y el valor de cada variable de shell

pwd: muestra la ruta de acceso del directorio actual

cd: cambia al directorio que se especifique, y en ausencia de éste cambia a /home

ls: muesta los nombres de los archivos del directorio actual por defecto, a menos que se especifique otro

df: muestra información sobre el sistema de ficheros

top: muestra la actividad en tiempo real del sistema

mkdir: crea uno o varios directorios si no existen

history: muestra la lista del historial de comandos con los números de línea

		\item ¿Para qué se usa tabulador en la terminal?
		\\R: Al presionarlo una vez autocompleta y dos veces muestra lista de opciones
		
		\item ¿Cuáles son los pasos para resolver un problema?
		\\R: Definir (entender); analizar y delimitar; buscar soluciones (es posible solucionarlo, qué soluciones se han dado, cuáles son las herramientas para hacerlo), y describir solución con detalle (en términos abstractos)
		
	\end{enumerate}
	
\end{document}





