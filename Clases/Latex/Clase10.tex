%\documentclass{article}
\documentclass[etterpaper, 12pt, oneside]{article}%formatea el documento
\usepackage{amsmath}
\usepackage{graphicx}
\usepackage{enumitem}
\usepackage{xcolor}
\usepackage[utf8]{inputenc}
\graphicspath{{Imágenes/}}

%Aquí inicia la portada del documento
\title{\Huge Taller de herramientas computacionales}
\author{Fernando Moctezuma Soto}
\date{18/enero/2019}

\begin{document}
	\maketitle
	%includegraphics[scale=0.40]{1.jpg}
	
	\newpage
	
	\title{\Huge Notas del curso\\}
	
	\textbf{Clase del 18 de enero.}\\
	
	Con ";" se separan comandos en el bash
	
	>>> type(cadena)
	<type 'str'>
	>>> H=type(cadena)
	>>> H
	<type 'str'>
	>>> H=type(cadena)._name_
	>>> H
	'str'
	
	C += iC equivale a C = C + iC
	
	Los resultados de expresiones lógicas son boolenaos
	
	Expresiones complejas como:
	-1 < a <= 0
	-1 < a and a <= 0
	not (-1 < a <= 0-1 < a)
	
	Las cadenas son objetos.
	Evaluación booleana de la cadena: debe contener caracteres para que sea True; ejemplo:
	>>> cadena="esta es mi cadena"
	>>> bool(cadena)
	True
	
	Las listas aparecen entre corchetes, sin no tiene caracteres es sin elementos.
	
	>>> L = []
	>>> if bool(L):
			print L
			
	>>> L = [1, 4, 6]
	>>> if bool(L):
			print L
			
	[1, 4, 6]
	
	
	Operaciones con listas:
	A través del método append se agregan elementos a la lista; ejemplo:
	>>> L.append(45)
	>>> L
	[1, 4, 6, 45]
	
	Con la función len() se conoce el número de elementos de una lista; ejemplo:
	>>> len(L)
	
	Nombre de la lista seguido del n-ésimo elemento de la lista entre corchetes muestra el objeto; ejemplo: 
	>>> L[45]
	
	También se ven los métodos: insert, pop, extend (requieren especificar índice)
	
	Ingresar y sacar elementos de una lista
	
	La posición es el índice más un elemento
	
	Investigar métodos relacionados con listas y guardar en:
	ListasS.py
	
	gradosC = [-20, -15, -10, -5, 0, 5, 10, 15, 20, 25, 30, 35, 40]
	
	>>> for C in gradosC:
			print 'Elemento de la lista: ', C
			
	>>> print 'La lista grados tiene', len(C), 'elementos'
	
	Se vió for, 
	range: crea una lista con los índices correspondientes a una lista, con un argumento,
	con dos: los índices desde el primer valor al ()n-1)-valor y con tres, el tercer valor es el incremento del primer al segundo.
	
	Cómo programar, Deitel
	
	A partir de gradosC = [-20]
	con while appen len hacer la lista de -20 a 30 con incremento de 2.5:
	
	while L[len(L)-1] !=30:
		L.append(L[len(L)-1]+2.5)
		
	hacer lista con flotante que avance desde el primer valor hasta el segundo valor, definida como
	rangeF(-30,31,20.5)
	guardado en utils
	
	
\end{document}
