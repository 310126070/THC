%\documentclass{article}
\documentclass[etterpaper, 12pt, oneside]{article}%formatea el documento
\usepackage{amsmath}
\usepackage{graphicx}
\usepackage{enumitem}
\usepackage{xcolor}
\usepackage[utf8]{inputenc}
\graphicspath{{Imagenes/}}

%Aquí inicia la portada del documento
\title{\Huge Taller de herramientas computacionales}
\author{Fernando Moctezuma Soto}
\date{20/enero/2019}

\begin{document}
	\maketitle
	
	\begin{center}
		\includegraphics[scale=0.40]{1.jpg}
	\end{center}
	
	\newpage
	
	\title{\Huge Notas del curso\\}
	
	\textbf{Clase del 18 de enero.}\\
	
	Con "\textbf{;}" se separan comandos en el bash.
	
	Las cadenas son objetos. La evaluación booleana de la cadena debe contener caracteres para que sea \emph{True}.
	
	Operaciones con listas:
	A través del método \textbf{append()} se agregan elementos a la lista.
	
	Con la función \textbf{len()} se conoce el número de elementos de una lista.
	
	El nombre de la lista seguido del n-ésimo elemento de la lista entre corchetes muestra el objeto.
	
	También se ven los métodos \emph{insert, pop, extend} cuya forma y papel son los siguientes:
	
	list.\textbf{insert}(i, x): inserta un elemento en una posición dada.
	
	list.\textbf{pop}([i]): elimina el elemento en la posición dada y lo devuelve.
	
	list.\textbf{extend}(L): amplía la lista agregando todos lo elementos de la lista dada.
	
	La posición es el índice más un elemento.
	
	\textbf{range}: crea una lista con los índices correspondientes a una lista, con un argumento; con dos, los índices desde el primer valor al (n-1)-ésimo valor, y con tres, el tercer valor que se incrementa del primero al segundo.
		
	Tarea: hacer lista con valores flotantes que avance desde el primer valor hasta el segundo valor, definida como rangeF(-30,31,20.5) y guardar como \emph{utils}.
	
	
\end{document}
