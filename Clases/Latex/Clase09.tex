%\documentclass{article}
\documentclass[etterpaper, 12pt, oneside]{article}%formatea el documento
\usepackage{amsmath}

\usepackage{graphicx}
\usepackage{enumitem}
\usepackage{xcolor}
\usepackage[utf8]{inputenc}
\graphicspath{{Imagenes/}}

%Aquí inicia la portada del documento
\title{\Huge Taller de herramientas computacionales}
\author{Fernando Moctezuma Soto}
\date{20/enero/2019}

\begin{document}
	\maketitle
	
	\begin{center}
		\includegraphics[scale=0.40]{1.jpg}
	\end{center}
	
	\newpage
	
	\title{\Huge Notas del curso\\}
	
	\textbf{Clase del 17 de enero.}\\
	
	
	Se vió el procedimiento para resolver el problema del MCD, en un programa de Python la función aparece definida como:
	
	\begin{verbatim}
	def MCD(a,b):
		if a<b:
			tmp=b
			b=a
			a=tmp
		r=a%b
		while r!=0:
			a=b
			b=r
			r=a%b
		return b
	\end{verbatim}
	
	Tras \textbf{if} se indican los pasos para encontrar el valor de la función y tras \textbf{while} están los pasos para llegar al valor buscado.
	
	Desde el \emph{shell} se pueden crear comandos y utilizarlos en la línea de comandos.
	
	
	

	
\end{document}
