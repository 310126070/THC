%\documentclass{article}
\documentclass[etterpaper, 12pt, oneside]{article}%formatea el documento
\usepackage{amsmath}
\usepackage{graphicx}
\usepackage{enumitem}
\usepackage{xcolor}
\usepackage[utf8]{inputenc}
\graphicspath{{Imágenes/}}

%Aquí inicia la portada del documento
\title{\Huge Taller de herramientas computacionales}
\author{Fernando Moctezuma Soto}
\date{16/enero/2019}

\begin{document}
	\maketitle
	%includegraphics[scale=0.40]{1.jpg}
	
	\newpage
	
	\title{\Huge Notas del curso\\}
	
	\textbf{Clase del 8 de enero.}\\
	
	Se comenzó con Git, que es un sistema de control de versiones gratuito y de código abierto. Los pasos en la configuración y los comandos de Git son:\\

	
	\textbf{Inicio.}
	
	Configuración de la información del usuario utilizada en todos los repositorios locales
	
	git config --global user.name:
	Establece un nombre que sea identificable cuando se revise el historial de versiones.
	
	git config --global user.email:
	Establecer una dirección de correo electrónico que se asociará con el de historial.\\
	
	\textbf{Cofiguración.}
	
	Configurando la información del usuario, inicializando y clonando repositorios.de confirmación
	
	git init:
	Inicializa un directorio existente como un repositorio Git.
	
	git clone:
	Recupera un repositorio completo desde una ubicación alojada a través de URL.\\
	
	\textbf{Instantáneas.}
	
	Instantáneas y preparación de Git.
	
	git status:
	muestra los archivos modificados en el directorio de trabajo.
	
	git add:
	Agrega un archivo a la próxima confirmación.
	
	git commit -m: 
	confirma el contenido como una nueva instantánea.\\
	
	\textbf{Actualización.}
	
	Recuperar actualizaciones de otro repositorio y actualizar repositorios locales.
	
	git push:
	Transmite datos locales, se sincroniza con la rama del repositorio remoto.
	
	git pull:
	Busca y combinar cualquier confirmación desde la rama remota de seguimiento.
	
\end{document}
