\documentclass[letterpaper, 12pt, oneside]{article}%especificaciones del documento
\usepackage{amsmath}%paquete para escribir expresiones matemámaticas
\usepackage{graphicx}%paquete para poder incluír imagenes en el documento
\usepackage{xcolor} %paquete de LaTex para poder poner otro texto
\graphicspath{{Imagenes/}}%directorio de la imagen, este lo cambian por el directorio en el que ustedes guardaron su imagen 1.png
\usepackage[utf8]{inputenc} %para poder poner acentos

	\title{\Huge Taller de Herramientas Computacionales}
	%\title{\Huge \colorbox{magenta}{Taller de Herramientas computacionales}} %De esta forma con colorbox pone el texto dentro de una "caja" de color.
	\author{Karla Adriana Esquivel Guzmán}%autor del escrito
	\date{11/01/19}%fecha del escrito

\begin{document}%inicia el documento
\maketitle
%\vfill %Para rellenar el espacio y colocar hasta abajo de la pagina el siguiente texto, imagen.
\begin{center}%inicia centrado
\includegraphics[scale=0.40]{1.png}
\end{center}%termina el centrado para la imagen
\newpage%crea una nueva página

\title{\Huge Primera Nota del curso\\}%titulo2 \\ sirven para saltar una linea.

Lo que hicimos en mi primera clase fue...%letras de color azul
\begin{enumerate}%Inicio de númeración para enlistar las cosas vistas en clase.
	\item Distribuciones de Linux: %item sirve enlistar el elemento, este es el primer elemento enumerado.
	\begin{enumerate}%Inicia otro enlistado dentro del listado anterior.
		\item Fedora
		\item Linux
	\end{enumerate}%finaliza el enlistado.
	\item Comandos de Bash%Segundo elemento enumerado.
	\begin{itemize}%comienza el enlistado pero itemize a diferencia de enumerate enlista sin un orden secuencial (es decir no utiliza números, ni letras)
		\item set
		\item cd
		\item mkdir
		\item mkdir -p
		\item ls
	\end{itemize}%finaliza enlistado con itemize
\end{enumerate}%finaliza el enlistado principal
\textbf{Este comando sirve para escribir en negritas}

\end{document}%termina el documento