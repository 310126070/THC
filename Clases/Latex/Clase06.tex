%\documentclass{article}
\documentclass[etterpaper, 12pt, oneside]{article}%formatea el documento
\usepackage{amsmath}
\usepackage{graphicx}
\usepackage{enumitem}
\usepackage{xcolor}
\usepackage[utf8]{inputenc}

%Aquí inicia la portada del documento
\title{\Huge Taller de herramientas computacionales}
\author{Fernando Moctezuma Soto}
\date{14/enero/19}

\begin{document}
	\maketitle
	
	\newpage
	
	\title{\Huge Clase 06\\}
	
	Se mostró una forma de resolver el problema de hallar el valor de la raíz de un número x; con una relación de la representación geométrica de un cuadrado de área x y un rectángulo de lados h=x y b=1. El problema se resulve al hallar los valores de b y h que se aproximen mediante las asignaciones:\\h=b+h/2 y b=x/h\\donde se cumple la igualdad\\ b*h=x.\\
	También se vieron las sentencias de control if, else y while. if y else ejecutan una acción si se cumple una condición y ejecutan otra si la misma no se cumple, y while, mientras se cumpla la condión, repiten ciertas acciones. Con estas sentencias y los operadores lógicos and y or se resolvió un problema de Diana.
	
	
	
	
\end{document}





