%\documentclass{article}
\documentclass[etterpaper, 12pt, oneside]{article}%formatea el documento
\usepackage{amsmath}

\usepackage{graphicx}
\usepackage{enumitem}
\usepackage{xcolor}
\usepackage[utf8]{inputenc}
\graphicspath{{Imagenes/}}

%Aquí inicia la portada del documento
\title{\Huge Taller de herramientas computacionales}
\author{Fernando Moctezuma Soto}
\date{20/enero/2019}

\begin{document}
	\maketitle
	
	\begin{center}
		\includegraphics[scale=0.40]{1.jpg}
	\end{center}
	
	\newpage
	
	\title{\Huge Notas del curso\\}
	
	\textbf{Clase del 16 de enero.}\\
	
	
	Funciones de algunos comandos en la terminal:
	
	\emph{Ctrl - z}: detiene ejecución de un programa.
	
	\emph{bg}: ejecuta programa en segundo plano.
	
	\emph{"nombre de programa" - \&}: ejecuta programa en segundo plano.
	
	\emph{Ctrl - c}: termina ejecución de un programa.
	
	\emph{gd} - "nombre de programa": regresa programa a primer plano.
	
	Dentro de top:
	
	\emph q: salir de top.
	
	\emph{kill - "señal" - "PID"}: para gestionar programas, por ejemplo, al sustituir "señal" por \emph{-a} y "PID" por el identificador de cierto proceso, este se termina.
	
	\textbf{./}...: Listado del directorio actual.
	
	\textbf{../}...: Listado del directorio padre.
	
	\emph{chmod +x - "nombre de programa"}: activa permisos de ejecución del programa.
	
	\emph{find . -name - "nombre de carpeta o archivo"}: busca carpeta o archivo en el directorio actual.
	
	\textbf{./} \emph{- "nombre de programa, seguido"}: se ejecuta programa en el bash.
	
	\emph{"nombre de lenguaje de programación" - "nombre de programa"}: otra forma de ejecutar desde bash un programa.\\
	
	En Python, con "\textbf{;}" se indica finalización de información.
	Existen clases, métodos, objetos y atributos; los métodos pueden modificar el estado del objeto o interactuar con él. Además, los métodos dependen del objeto, mientras que la función no. Con "\emph{type()}" se conoce la clase a la que pertenece el objeto cuyo nombre aparece como argumento.
	
	
	

	
\end{document}
