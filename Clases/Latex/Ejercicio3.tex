\documentclass{book}
\usepackage[spanish]{babel}
\usepackage[utf8]{inputenc}
\usepackage{biblatex}
\usepackage{hyperref}


\title{Taller de herramientas computacionales}
\author{Fernando Moctezuma Soto}
\date{17/enero/2019}

\begin{document}
	
	\maketitle
	%Aquí inicia el índice de contenido del texto
	\tableofcontents
	\section*{Introducción} Este libro es para fortalecer el conocimiento de la materia de THC.
	\url{www.google.com}
	\hyperref[Google]{www.google.com}
	
	\chapter{Uso básico de GNU/Linux}
	\section{Distribuciones de Linux}
	\section{Comandos}
	\section{Introducción a Latex}
	\section{Introducción a Python}
	
	
	\begin{verbatim}
		#!/usr/bin/python2.7
		# -*- coding: utf-8 -*-
		
		'''
		Moctezuma Soto, Fernando
		310126070
		Taller de herramientas computacionales
		
		Programa que imprime "Hoy es miércoles"
		'''
		
		
		x = 10.5;  y = 1.0/3;  z = 15.3
		# x, y, z = 10.5, 1.0/3, 15.3
		
		H = """El punto en R3 es:
		(x, y, z) = (%.2f, %g, %G)
		""" % (x, y, z)
		print H
		
		
		G = """
		El punto en R3 es:
		(x, y, z) = ({laX:.2f}, {laY:g}, {laZ:G})
		""" .format(laX=x, laY=y, laZ=z)
		print G
		
		
		
		import math as m
		from math import sqrt 
		#from math import *
		
		x=16
		x=input("¿Cuál es el valor al que le quieres calcular la raiz?: ")
		print "La raíz cuadrada de %.2f es %f" % (x, m.sqrt(x))
		print sqrt(16.5)
	\end{verbatim}
	
	
	\input{prueba.py}
	
	
	%Aquí inician los capitulos del libro
	\chapter{Introducción a Latex}
	\chapter{Introducción a Python}
	\section*{Orientación a objetos}
	
	\begin{thebibliography}{9}
		\bibitem{Libro}
		Autor bla bla bla\\
		\textit{Cualquier cosa}
		bla bla bla 2019
	\end{thebibliography}
	
	
\end{document}