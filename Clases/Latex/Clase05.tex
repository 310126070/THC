%\documentclass{article}
\documentclass[etterpaper, 12pt, oneside]{article}%formatea el documento
\usepackage{amsmath}

\usepackage{graphicx}
\usepackage{enumitem}
\usepackage{xcolor}
\usepackage[utf8]{inputenc}
\graphicspath{{Imagenes/}}

%Aquí inicia la portada del documento
\title{\Huge Taller de herramientas computacionales}
\author{Fernando Moctezuma Soto}
\date{20/enero/2019}

\begin{document}
	\maketitle
	
	\begin{center}
		\includegraphics[scale=0.40]{1.jpg}
	\end{center}
	
	\newpage
	
	\title{\Huge Notas del curso\\}
	
	\textbf{Clase del 11 de enero.}\\
	
	
	En Python con una comilla o un par de comillas se introduce una cadena, que es una secuencia de caracteres, y con tres comillas se indica un comentario multilínea. Se puede utilizar \emph{print} y tres comillas para mostrar texto cuando el programa se ejecuta. Con una sola comilla y \emph{/n} se salta de línea, después de \emph{print}.
	
	\%g indica el lugar donde se sustituye el valor de una variable, para dar formato a la salida del texto; con \%e o \%E el valor será en formato científico y con \%s se sustituye una variable que contenga una cadena.
	
	Un \emph{script} es un conjunto de instrucciones de determinado lenguaje interpretado.
	
	Un módulo es una biblioteca, en la que están definidas funciones. Al crear un módulo, este se importa con el nombre del archivo en donde se encuentra.


	
	

	

	
\end{document}
